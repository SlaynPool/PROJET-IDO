
\documentclass[10pt,a4paper]{article}
\usepackage[utf8]{inputenc}
\usepackage[french]{babel}
\usepackage[left=2cm,right=2cm,top=2cm,bottom=2cm]{geometry}
\usepackage{hyperref}
\usepackage{graphicx}

%opening
\title{Projet}
\author{Nicolas Vadkerti}
\usepackage{listings} % Required for inserting code snippets
\usepackage[usenames,dvipsnames]{color} % Required for specifying custom colors and referring to colors by name

\definecolor{DarkGreen}{rgb}{0.0,0.4,0.0} % Comment color
\definecolor{highlight}{RGB}{255,251,204} % Code highlight color

\lstdefinestyle{Style1}{ % Define a style for your code snippet, multiple definitions can be made if, for example, you wish to insert multiple code snippets using different programming languages into one document
language=Perl, % Detects keywords, comments, strings, functions, etc for the language specified
backgroundcolor=\color{highlight}, % Set the background color for the snippet - useful for highlighting
basicstyle=\footnotesize\ttfamily, % The default font size and style of the code
breakatwhitespace=false, % If true, only allows line breaks at white space
breaklines=true, % Automatic line breaking (prevents code from protruding outside the box)
captionpos=b, % Sets the caption position: b for bottom; t for top
commentstyle=\usefont{T1}{pcr}{m}{sl}\color{DarkGreen}, % Style of comments within the code - dark green courier font
deletekeywords={}, % If you want to delete any keywords from the current language separate them by commas
%escapeinside={\%}, % This allows you to escape to LaTeX using the character in the bracket
firstnumber=1, % Line numbers begin at line 1
frame=single, % Frame around the code box, value can be: none, leftline, topline, bottomline, lines, single, shadowbox
frameround=tttt, % Rounds the corners of the frame for the top left, top right, bottom left and bottom right positions
keywordstyle=\color{Blue}\bf, % Functions are bold and blue
morekeywords={}, % Add any functions no included by default here separated by commas
numbers=left, % Location of line numbers, can take the values of: none, left, right
numbersep=10pt, % Distance of line numbers from the code box
numberstyle=\tiny\color{Gray}, % Style used for line numbers
rulecolor=\color{black}, % Frame border color
showstringspaces=false, % Don't put marks in string spaces
showtabs=false, % Display tabs in the code as lines
stepnumber=5, % The step distance between line numbers, i.e. how often will lines be numbered
stringstyle=\color{Purple}, % Strings are purple
tabsize=2
}

\newcommand{\insertcode}[2]{\begin{itemize}\item[]\lstinputlisting[caption=#2,label=#1,style=Style1]{#1}\end{itemize}} 


% \insertcode{"Scripts/example.pl"}{Nena would be proud.} 

\begin{document}

\maketitle


\url{https://github.com/SlaynPool/PROJET_IDO/}

\section{Drone}
Le but de ce document est de décrire les etapes pour la realisation de mon projet.
En effet, le projet sera découpé en 2 parties majeurs.
\begin{itemize}
 \item Conception de la Power board distribution
 \item Programmation de la FC
 
\end{itemize}
\section{Conception de la Power board distribution}
Pour cela, il faut etre consient des differents parties d'un drone.
Celui-ci est, comme on peut ce douter composé de 4 moteurs ou plus. J'ai choisis de créer un drone avec 4 bras, car la documentation technique est plus abondante sur celui-ci. \\
Les moteurs ne sont pas pilotés directement par le controleur de vol. En effet, les moteurs couraments utilisés dans le modélisme sont des moteurs dit ``brushless``. Ils sont bien plus pratique à utilisés comparé aux moteurs ''Brushed``, ou dit moteurs à charbon. Les moteurs brushless sont plus petit, offre plus de couple, et sont donc plus efficase. Cependant, comme leurs fonctionnements est beaucoup plus complexe, il faudra utilisés des ESC, electronic speed control. C'est eux qui se changeront de controler la vitesse des moteurs. La plupart des ESC on un fonctionnements simples. Il suffit de la piloter grâce à un simple PWM.
Les ESC devront etre relié à la batterie pour faire tourner les moteurs, et ils devront etre connecter au controleur de vol.\\
Le controleur de vol sera soit un Arduino, ou un STM32. Il faudra donc prevoir sur la PDB un endroit où on pourra récupérer du 9V pour l'arduino par exemple
L'idée sera donc d'obtenir 


 
 
 
 




\end{document}

