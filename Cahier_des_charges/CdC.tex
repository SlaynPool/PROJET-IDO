\documentclass{article}
\usepackage[utf8]{inputenc}

\title{Projet Drone}
\author{nicolas.vadkerti }
\date{October 2019}

\begin{document}

\maketitle
\section{Contexte}
Nous voulons faire voler un drone avec pour objectif de faire de la video stéréoscopique. Nous ne nous occuperons pas de la partie Traitement De L'image. Notre rôle sera de fournir l'outil pour crée le contenu. Cela inclu la transmision video en temps réels, ainsi que toutes les metadatas liés au vol en temps réel. Toutes les infos récoltées pourrons permettre en parallèle pour du vol coordonée voir meme de la redondance (exemple : un drone perd le signal radio, le second drone redonde le signal )  

\section{Concevoir un drone}
\subsection{Doit être stable}
\begin{itemize}
    \item Etude des formes possibles (quad,tri,Hexa)
    \item Etude du Poids (Moins de 25 Kilo si on veut rester légal) Ne pas oublier: Plus c'est lourd plus ca consomme 
    \item Ajout de capteurs divers 
    \item Compréhension de la notion de PID
\end{itemize}
\subsection{Etude/Choix d'un firmware Libre} https://opensource.com/article/18/2/drone-projects
\subsection{Navigation}
\begin{itemize}
    \item Version Classique (RX-TX)?
    \item Navigation Automatique Via GPS ?
    \item Pilotage via PC ? 
\end{itemize}

\section{Partie Connecté}
\subsection{Transmision Video}
Possibilité: \\
\begin{itemize}
    \item Analogique Classique  :  VTX 
    \subitem Si l'on veut lier la video avec les metadatas, fraudra avoir des timestamps (supposition)
    \item Systeme Perso ? 
    
\end{itemize}
\subsection{Creation/Transmision des metadatas}
\begin{itemize}
    \item Comment on transmet ?
    \item On fais une carte de vol et une carte avec les capteurs non nessecaires au vol / ou on fait une carte de vol All-in ONE (On la Conçoit ?)
    \item Interpretation des données ? (https://github.com/cleanflight/blackbox-tools) BDD/GRAPHANA ?
\end{itemize}







\end{document}
