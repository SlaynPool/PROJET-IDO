\documentclass{article}

% USEPACKAGE
\usepackage[utf8]{inputenc}
\usepackage{pgfgantt}
\usepackage{tikz}
\usepackage{lscape}
\usepackage{amsmath}
% end USEPACKAGE
\title{Projet Drone}
\author{nicolas.vadkerti }
\date{October 2019}

\begin{document}

\maketitle
\section{Contexte}
Nous voulons faire voler un drone avec pour objectif de faire de la video stéréoscopique. Nous ne nous occuperons pas de la partie Traitement De L'image. Notre rôle sera de fournir l'outil pour crée le contenu. Cela inclu la transmision video en temps réels, ainsi que toutes les metadatas liés au vol en temps réel. Toutes les infos récoltées pourrons permettre en parallèle pour du vol coordonée voir meme de la redondance (exemple : un drone perd le signal radio, le second drone redonde le signal )  

\section{Concevoir un drone}
\subsection{Doit être stable}
\begin{itemize}
    \item Etude des formes possibles (quad,tri,Hexa)
    \item Etude du Poids (Moins de 25 Kilo si on veut rester légal) Ne pas oublier: Plus c'est lourd plus ca consomme 
    \item Ajout de capteurs divers 
    \item Compréhension de la notion de PID (https://journals.sagepub.com/doi/full/10.5772/53747)
\end{itemize}
\subsection{Etude/Choix d'un firmware Libre} https://opensource.com/article/18/2/drone-projects
\subsection{Navigation}
\begin{itemize}
    \item Version Classique (RX-TX)?
    \item Navigation Automatique Via GPS ?
    \item Pilotage via PC ? 
\end{itemize}

\section{Partie Connecté}
\subsection{Transmision Video}
Possibilité: \\
\begin{itemize}
    \item Analogique Classique  :  VTX 
    \subitem Si l'on veut lier la video avec les metadatas, fraudra avoir des timestamps (supposition)
    \item Systeme Perso ? 
    
\end{itemize}
\subsection{Creation/Transmision des metadatas}
\begin{itemize}
    \item Comment on transmet ?
    \item On fais une carte de vol et une carte avec les capteurs non nessecaires au vol / ou on fait une carte de vol All-in ONE (On la Conçoit ?)
    \item Interpretation des données ? (https://github.com/cleanflight/blackbox-tools) BDD/GRAPHANA ?
\end{itemize}

\begin{landscape}
\section{Découpage des taches}
% \begin{ganttchart}{1}{30}
%  \gantttitle{test}{30}\\
%  \gantttitlelist{1,...,30}{1}\\
%  \ganttbar{Etude des possibilées}{1}{2}
  % \end{ganttchart}




%Used to draw gantt charts, which I will use for the calendar.
%Let's define some awesome new ganttchart elements:
\newganttchartelement{orangebar}{
    orangebar/.style={
        inner sep=0pt,
        draw=red!66!black,
        very thick,
        top color=white,
        bottom color=orange!80
    },
    orangebar label font=\slshape,
    orangebar left shift=.1,
    orangebar right shift=-.1
}

\newganttchartelement{bluebar}{
    bluebar/.style={
        inner sep=0pt,
        draw=purple!44!black,
        very thick,
        top color=white,
        bottom color=blue!80
    },
    bluebar label font=\slshape,
    bluebar left shift=.1,
    bluebar right shift=-.1
}

\newganttchartelement{greenbar}{
    greenbar/.style={
        inner sep=0pt,
        draw=green!50!black,
        very thick,
        top color=white,
        bottom color=green!80
    },
    greenbar label font=\slshape,
    greenbar left shift=.1,
    greenbar right shift=-.1
}


\begin{ganttchart}[
    hgrid style/.style={black, dotted},
    vgrid={*5{black,dotted}, *1{white, dotted}, *1{black, dashed}},
    x unit=3mm,
    y unit chart=9mm,
    y unit title=12mm,
    time slot format=isodate,
    group label font=\bfseries \Large,
    link/.style={->, thick}
    ]{2019-10-24}{2019-12-21}
    \gantttitlecalendar{year, month=name, week}\\

    \ganttgroup[
        group/.append style={fill=orange}
    ]{Etude des solutions}{2019-10-24}{2019-11-10}\\ [grid]
    \ganttorangebar[
        name=Documentation
    ]{Documentation}{2019-11-4}{2019-11-10}\\ [grid]
    
    \ganttorangebar[
        name=Completion du Diagramme
    ]{Completion du Diagramme}{2019-11-11}{2019-11-17}
%     \ganttlinkedorangebar{}{2019-10-7}{2019-10-10}
%     \ganttlinkedorangebar{}{2019-10-14}{2019-10-15}
%     \ganttlinkedorangebar{}{2019-10-17}{2019-10-17}
%     \ganttlinkedorangebar[name=FMEend]{}{2019-10-21}{2019-10-24}
%     \ganttlinkedorangebar{}{2019-10-28}{2019-10-31}\\ [grid]
%     \ganttorangebar[name=Manual]{Manual}{2019-10-30}{2019-10-31}
%     \ganttlinkedorangebar{}{2019-11-4}{2019-11-7} \ganttnewline[thick, black]
% 
%     \ganttgroup[
%         group/.append style={fill=blue}
%     ]{Test Cases}{2019-10-27}{2019-11-9}
%     \ganttgroup[
%         group/.append style={fill=blue}
%     ]{}{2019-11-17}{2019-12-19}\\ [grid]
%     \ganttbluebar{Innocent testing}{2019-10-30}{2019-10-31}
%     \ganttlinkedbluebar[name=Innocent]{}{2019-11-4}{2019-11-7}
%     \ganttlinkedbluebar{}{2019-12-4}{2019-12-5}
%     \ganttlinkedbluebar{}{2019-12-9}{2019-12-10}\\ [grid]
%     \ganttbluebar{Test Case Testing}{2019-11-6}{2019-11-7}
% 
%     \ganttbluebar[name=Writing]{Writing}{2019-11-18}{2019-11-19}
% 
%     \ganttgroup[
%         group/.append style={fill=green}
%     ]{KLIP Manager}{2019-11-3}{2019-11-9}
%     \ganttgroup[
%         group/.append style={fill=green}
%     ]{}{2019-11-17}{2019-11-23}\\ [grid]
%     \ganttgreenbar{Manual}{2019-11-4}{2019-11-7}
%     \ganttlinkedgreenbar{}{2019-11-18}{2019-11-19}

    %Implementing links
%     \ganttlink[link mid=0.75]{Documentation}{FME}
%     \ganttlink{FMETutorial}{FME}
\end{ganttchart}

\end{landscape}
\section{Etude des solutions}
\subsection{Forme du drone}
Un drone peut etre formé de differentes facons. Il peut etre composé de 1 à n bras. Le choix du nombre de bras ce fait notament pour des problemes de protances. En effet, plus on a d'hélice, plus on va pouvoir soulever du poid 

Source : https://tel.archives-ouvertes.fr/file/index/docid/57385/filename/these.pdf Intersant à partir page 57



\end{document}
